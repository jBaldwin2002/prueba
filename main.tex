\documentclass[12pt, Letterpaper]{article}
\usepackage[utf8]{inputenc}
\usepackage{biblatex}
\usepackage{makecell}
\usepackage{fancyhdr}
\usepackage{graphics}
\graphicspath{{img/}}
\usepackage{amsmath}
\usepackage{multicol}
\usepackage{graphicx}
\usepackage{cancel}
\usepackage{amssymb}
\usepackage{todonotes}
\usepackage{ dsfont }
\usepackage{anyfontsize}
\usepackage{pdflscape}


\begin{document} 
\begin{center}

\includegraphics[scale=0.25]{Logo.png.jpg}\\[1cm]
\vspace{1.2cm}

{\LARGE Robótica y Automatización}\\
 

\vspace{2,5cm}

{\Large Baldwin Palencia Serrano}\\

\vspace{2cm}

{\Large Profesor\\Wilmer Mesias López López}
\vspace{2.5cm}

{\large UNIVERSIDAD CENTRAL\\ 
INGENIERÍA DE SISTEMAS/PRÁCTICA INGENIERÍA I\\
BOGOTÁ D.C\\ 
2023}

\vspace{1.7cm}

\end{center}
\begin{landscape}
     \begin{center}
         {\Huge Definición}
     \end{center} 

{\Large La robótica se ocupa del diseño, construcción y operación de robots, que son máquinas programables capaces de realizar tareas de forma autónoma o semiautónoma. Los robots pueden ser utilizados en una amplia variedad de aplicaciones, desde la fabricación y la construcción hasta la exploración espacial y la medicina.} 

\vspace{0.5cm}

\section{Estado Actual de la Robótica}

\begin{itemize}
    \item {\large Madurez tecnológica}
\end{itemize}

\begin{center}
    \includegraphics[scale=0.47]{imaguen_1.png}\\
\end{center}
\vspace{0.0cm}


 \section{Elementos básicos}
\begin{itemize}

    \item {\large Manipulador}
\begin{center}
    \includegraphics[scale=0.80]{manipulador _1.png}\\
\end{center}
\vspace{5.0cm}
    
    \item {\large Accionamientos}
    \begin{center}
    \includegraphics[scale=0.25]{Accionamientos.png}\\
\end{center}
\vspace{0.0cm}

    \item {\large Elementos terminales}
    \begin{center}
    \includegraphics[scale=0.45]{Elementos terminales_1.png}\\
\end{center}
\vspace{0.0cm}
    
    \item {\large Periferia del robot}
\end{itemize}


\section{Programación y simulación de robots}
\begin{itemize}
    \item {\large Programación por aprendizaje}
    \item {\large Aprendizaje activo}
    \item {\large Aprendizaje pasivo}
    \item {\large Programación textual}
    
\end{itemize}


 \section{Algunos beneficios de la Robótica y Automatización}
\begin{itemize}
    \item {\large Reducción de costos y riesgos operativos}
    \item {\large Flexibilidad y simplicidad}
    \item {\large Análisis y Métricas más precisas}
    \item {\large Mayor productividad}
    \item {\large Liberación de tareas monótonas}
    \item {\large Mayor compromiso de los empleados}
    \item {\large Invariabilidad de la calidad}
\end{itemize}


\vspace{1cm}

\begin{center}
 
         {\Huge Referencias}
     \end{center} 

     \begin{itemize}
         \item {\Large Juan Carlos Hernández Matías y Antonio Vizán Idoipe.(2015). Sistemas de automatización y robótica para las pymes españolas. Escuela de organización industrial}

\vspace{0.7cm}

      \item {\Large Eduardo }
   \end{itemize}

\end{landscape}
\end{document}
